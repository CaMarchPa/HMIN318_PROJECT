\par\hspace{0.5cm} \lettrine[lines=2]{\textbf{L}} {}a segmentation est un processus crucial dans l'analyse et le traitement d'images(2D/3D). Elle conditionne par ailleurs tous les calculs qui la suivent. La segmentation a pour but de séparer l'image en des régions dont les voxels (pixels pour des images 2D) présentent des caractéristiques semblables. Chaque région est supposée correspondre à un objet de l'image. Il existe plusieurs méthodes de segmentation d'images. La méthode utilisée dépend donc très fortement du type d'images ainsi que de l'application visée.

En imagerie médicale, les images ont des caractéristiques particulières. En effet, les niveaux de gris sont liés aux caractéristiques physiques des tissus.
Dans ce projet, je me focalise sur la segmentation des images CT de Sinus. Une image CT (\textbf{CT}: Computerized Tomography) est une technique d'acquisition d'images, qui consiste à l'absorption des rayons X par les tissus. Le sinus est une cavité dans un tissu du corps humain qui est le plus souvent osseux.

L'objectif principal de ce projet est de mettre en œuvre un outil automatique permettant de réaliser la segmentation des sinus dans des images CT par seuillage local.

Dans la suite de ce rapport, Je vais faire un descriptif des images CT des Sinus. Puis, je fais une brève étude bibliographique pour comprendre les principes. Ensuite, je présente quelques approches de segmentation que j'ai implémenté. En fin, je présente les résultats que j'ai obtenu, qui feront l'objet d'une conclusion et de perspectives.