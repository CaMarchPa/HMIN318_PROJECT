La tomodensiométrie permet d'obtenir des images CT de \textbf{\textit{Sinus}} automatiquement. En revanche, la segmentation automatique des images CT, notamment des sinus est limitée à cause du bruit et de l'anatomie complexe des sinus. En effet, ils existent plusieurs approches dans la littérature pour la segmentation des sinus. Par exemple, une méthode qui consiste à utiliser un seuil et des opérations morphologiques automatiques pour estimer une région d'intérêt et un contour initial. La région basée sur le contour initialement obtenu est ensuite appliquée pour extraire la frontière désirée\cite{ref1} et permet de distinguer les os des sinus(cavités vides). Une autre méthode utilise un système de coordonnées à partir de la longueur, de la largeur et de la profondeur de l'image\cite{ref0}. La méthode de segmentation proposée par \cite{ref0} est basée sur les caractéristiques d'intensité dans les sinus, les caractéristiques géométriques communes des sinus observée et les particularités des sites d'ouvertures paranasaux aux voies nasales.

Ces deux méthodes susmentionnées sont différentes et donnent des résultats assez intéressants. La plus différence entre ces deux méthodes est que \cite{ref1} cherche de contours pour envelopper des régions d'intérêts, tandis que \cite{ref0} met l'accent sur les caractéristiques d'intensités géométriques des sinus observées.